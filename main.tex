\documentclass{article}
\usepackage[utf8]{inputenc}
\usepackage{geometry}
 \geometry{
 a4paper,
 total={170mm,257mm},
 left=20mm,
 top=20mm,
 }
 \usepackage{graphicx}
 \usepackage{titling}

 \title{Leveraging Blockchain for Enhanced IoT Security
}
\author{Srivatsan Srinivasan}
\date{January 2023}
 
 \usepackage{fancyhdr}
\fancypagestyle{plain}{%  the preset of fancyhdr 
    \fancyhf{} % clear all header and footer fields
    \fancyfoot[R]{\includegraphics[width=4cm]{snulogo.jpg}}
    \fancyfoot[L]{\thedate}
    \fancyhead[L]{Summary of the Research Paper}
    \fancyhead[R]{\theauthor}
}
\makeatletter
\def\@maketitle{%
  \newpage
  \null
  \vskip 1em%
  \begin{center}%
  \let \footnote \thanks
    {\LARGE \@title \par}%
    \vskip 1em%
    %{\large \@date}%
  \end{center}%
  \par
  \vskip 1em}
\makeatother

\usepackage{lipsum}  
\usepackage{cmbright}

\begin{document}

\maketitle 

\noindent
\normalsize
Submission Made by:

\begin{tabular}{@{}ll}
    Name & \ Srivatsan Srinivasan\\
     Registration Number &  21011101129\\
     Class & AI-DS B
\end{tabular}

\section*{Topic Summary}
\large The Internet of Things (IoT) is a rapidly growing field that has the potential to change the way we live and work. It involves connecting billions of devices, such as smartphones, appliances, vehicles, and industrial equipment, to the internet to allow them to communicate with each other. This interconnected network of devices can provide a wide range of benefits, such as improved efficiency, automation, and data collection. However, with the increasing number of connected devices, the security risks associated with IoT networks have also grown.

One of the biggest challenges with securing IoT devices is the fact that many of them are resource-constrained and lack the computational power or memory to run traditional security solutions. Additionally, many IoT devices are designed to operate in environments that are difficult to secure, such as remote locations or in the field. This makes them vulnerable to a wide range of security threats, such as hacking, data breaches, and malware infections.

By leveraging blockchain, it is possible to create secure and tamper-proof systems for authenticating and managing IoT devices. Blockchain can also be used to secure the data generated by IoT devices, by storing it on a decentralized and tamper-proof ledger. This makes it much more difficult for hackers to access or alter the data, as they would need to compromise multiple nodes on the network. Additionally, blockchain can be used to create secure communication channels between IoT devices, which can further enhance the security of the overall system.

In summary, Blockchain technology offers a powerful solution to the security challenges faced by the Internet of Things. By leveraging its decentralized and secure nature, it is possible to create tamper-proof systems for authenticating, managing, and securing IoT devices and their data. This makes it an attractive option for organizations looking to deploy IoT solutions and secure them from potential threats.

\section*{Key Contributions from the Author}
\begin{itemize}
    \item BlockChain ideas and concepts: Blockchains use cryptography to connect records or ‘pieces’, with every piece containing a timestamp, the cryptographic hash of the previous block and the information or ‘Merkle tree’ to be exchanged. The information contained in a blockchain cannot be adjusted, can only be recorded once and cannot be changed retrospectively
        \begin{itemize}
        \item Block Body: Relies on 4 main features: 
        \begin{itemize}
            \item Ledger: a technicality employment an add as it were recorded to equipping full value-based history. Not at all like traditional databases, are exchanges while values at a blockchain not been overruns.
            \item Secure: Blockchains are encrypted, thereby ensuring that the information stored therein cannot be manipulated without authorisation.
            \item Shared: Each record involves numerous members. This is gives straightforwardness over a hub member within a blockchain web.
            \item Distributed: A blockchain may be disseminated, and the number of nodes in a blockchain can be calibrated to increase its flexibility and to reduce the possibility of a successful attack.  
        \end{itemize}
    
        \item Characterestics of BlockChain Tech: Whilst built on sound and largely         understood cryptographic standards, the use of blockchain remains at its infancy. This technology is partly surrounded by build-up, to which several solutions have been proposed. Moving fore, it is probable while the buildup shall pass on down, and blockchain technology shall ended up adjuster another tool that can be used.
    
        \item Applications of BlockChain in fields of :
            \begin{itemize}
                \item IOT
                \item Security
                \item Finance
                \item Reputation System
                \item Public Service
            \end{itemize}
    
        \end{itemize}

    \item IoT : IoT collectively refers to seamlessly connected devices that communicate with one another via the Internet. The increasing number of modern-day devices that are connected to the Internet, including digital assistants (e.g. Alexa and Siri), refrigerators, measurement sensors and lighting equipment, continuously blurs the boundary between Wi-Fi-enabled devices and other devices 
        \begin{itemize}
            \item Components of IoT: 
            \begin{itemize}
                \item Sensor/devices: Sensors or devices are basic elements of an IoT that collect data from devices with an Internet protocol (IP) address. These devices may be as simple as temperature/humidity sensors installed in buildings or as complex as intelligent vehicles.

                \item Connectivity: Sensor and other devices are connected to a cloud by using various types of networks, including LAN, Wi-Fi, satellite, Bluetooth and cellular infrastructure. Therefore, IoT devices utilise standard communication protocols to communicate with one another.

                \item Data processing: After being stored in the cloud, the data are processed with the help of software. Before they can be of any use, these collected data should be processed, filtered and analysed in a specialised manner. To prevent the system from being overwhelmed, zettabytes of data are guided thru each edge gateway before they can be processed. In this example, IoT is used to collect information from real-world environments through sensors in order to formulate real-time agile decisions.

                \item User interface (UI): After being cleaned and formatted, the collected data should also be used to alert end users. For example, users need to be alerted about the temperature in cold storage. A UI serves this purpose by allowing end users to proactively check the collected data. Therefore, these end users may also react to the system inputs depending on the IoT applications.
            \end{itemize}

            \item Data Privacy and Sovereignty of IoT: 
            Data privacy and self-sovereign identity are two important considerations in the context of the Internet of Things (IoT). As more and more devices are connected to the internet, the amount of data generated by these devices is also increasing. This data can be highly sensitive and personal, such as location data, health information, or financial transactions. Ensuring that this data is protected from unauthorized access or use is critical to maintaining the privacy and security of individuals.

            Self-sovereign identity refers to the ability of individuals to control and manage their own digital identity, rather than having it controlled by a centralized authority. This is particularly relevant in the context of IoT, where individuals may have multiple devices that are generating and transmitting data about them. By using self-sovereign identity, individuals can ensure that their data is only shared with authorized parties and that they have control over how it is used. Additionally, it can also help to prevent identity fraud and other malicious activities.

            \item PEFM: PEFM is an indirect policy mechanism applied in IoT to ensure the data privacy of users regardless of the sensitivity of such data. This policy essentially seals the network, hence effectively blocking the unauthorised entry of data and preventing the data of users from falling into the wrong hands. PEFM is a policy technology that utilises security
            mechanisms to ensure that none of the information stored by legal users in the network will be leaked. This mechanism depends on the creation and activation of data and utilises a virtual machine. PEFM is not an executable machine; rather, this operating mechanism is embedded into IoT to avoid the risk of privacy attacks. PEFM employs two methods, namely,
            apoptosis and evaporation.
        \end{itemize}

    \item BlockChain Techonology Based IoT: Blockchain-based Internet of Things (IoT) refers to the use of blockchain technology to secure and manage IoT networks and devices. Blockchain is a decentralized and secure technology that enables the creation of tamper-proof digital records. It can be used to authenticate and manage IoT devices, secure the data they generate and transmit, and create secure communication channels between devices. This makes it an attractive solution for addressing the security and privacy challenges faced by the IoT. Additionally, blockchain can also be used to create self-sovereign identity which allows individuals to control and manage their own digital identity and prevent identity fraud. Leveraging blockchain technology can provide enhanced security, transparency, and efficiency to IoT systems, making it an important area of research and development.


    \item IoT Use Scenarios: 

    \begin{itemize}
        \item Supply chains: A supply chain usually involves multiple users operating in different time zones, hence contributing to its complexity.The importance of using blockchains in these supply chains has been recognised in recent years by food distributors, pharmaceutical companies, seaport operators and electronics manufacturers amongst others.

        \item Smart appliances: Smart home equipment is predicted to be a staple in every home in the future. Specifically, new houses and structuresare predicted to have several features that can be triggered using smart devices or the Internet. By warding off statisticsstorage on the predominant server or in cloud storage in choose of blockchain, private statistics can be blanketed and the domestic IoT community might also be secured.

        \item Electricity markets: Additionally, statistics from clever units can be used to improve power consumption in the community than at thehousehold or individual level. This option has already received the attention of governments and enterprises. For instance, Siemens partnered with Energy a few years ago to establish a smart grid that utilises blockchain in buying and selling electricity.

        \item Transport: The transportation industry is currently experimenting with the use of blockchains in IoT, such as in tracing the origins of automobiles and their spare parts. For instance, two of the greatest automobile manufactures in the world, Jaguar Land Rover and Volkswagen, started exploring smart contracts with the aid of IOTA. Specifically, a smart pocket is installed in vehicles, which would allow vehicle operators to earn IOTA tokens in exchange for data on road conditions and congestion and even join carpooling programmes. They may also utilise such tokens to pay for their parking or to charge their electric vehicles. 

        \item Smart Locks: Slock, a German company, introduced the Smart Lock technology that uses self-reliant keys in order for to rent, share or sell a wide range of commodities, including homes or flats, bicycles and automobiles.

    \end{itemize}
\end{itemize}

\section*{My Views on the Topic at hand}
The Internet of Things (IoT) and blockchain technology have the potential to bring many benefits to humanity, but they also present some challenges and potential negative consequences.

On the positive side, IoT and blockchain can be used together to create more efficient, secure and transparent systems for various industries such as supply chain, healthcare, and energy management. They can also be used to create self-sovereign identities, providing individuals more control over their own personal data. Additionally, the combination of IoT and blockchain can improve transparency and security in financial transactions, reducing the risk of fraud and increasing trust in the system.

However, there are also some potential downsides to the widespread adoption of IoT and blockchain. One of the main concerns is the issue of data privacy and security. As more and more devices are connected to the internet, the amount of data being generated and transmitted is also increasing. This data can be highly sensitive and personal, and if it falls into the wrong hands, it can be used for malicious purposes. Additionally, many IoT devices are resource-constrained and lack the computational power or memory to run traditional security solutions, making them vulnerable to hacking and other security threats.

Another concern is the energy consumption of blockchain, as it requires a lot of computational power to maintain the network, which can lead to a significant increase in carbon footprint.

As a conclusion to my trail of thoughts, IoT and blockchain have the potential to bring many benefits to humanity, but it's important to consider the potential downsides and take steps to mitigate them. This includes ensuring the security and privacy of data, and addressing the energy consumption issues.

\section*{Agreements, Pitfalls and Fallacies}

\begin{itemize}
    \item The Author states "With their continuous growth, enterprises need to deploy IoT to store their information. IoT based on blockchains has penetrated virtually all aspects of our daily lives, and some consumers are not even aware that they are actively using such technology.", and in my opinion, this is something that I agree with unequivocally. As presented in this paper itself, in 2008, the number of devices connected to the Internet exceeded the number of humans on Earth. And in the case of 2023, it's needless to say it's now increased multiple folds of the current population as well. The Internet of Things (IoT) has become an integral part of our daily lives, and its impact is only set to grow in the future. From smart homes and appliances to connected cars and wearables, IoT devices are everywhere. They make our lives more convenient, efficient, and connected. We can control the temperature of our homes from our smartphones, track our fitness goals with wearable devices, and even have our groceries delivered to us by autonomous vehicles. The amount of data generated by these devices also provides valuable insights for businesses, allowing them to improve products and services and make more informed decisions. Overall, IoT has the potential to greatly enhance our quality of life and improve the way we live, work and interact with each other.

    \item The Author also dedicates a section about how Data Security is upheld when Blockchain technology is deployed in IoT Systems, However, in my opinion While blockchain technology can provide enhanced security for IoT networks and devices, there are also some concerns about the risks it poses when it comes to data security.
    One of the main concerns is the issue of data privacy. As more and more data is generated and stored on blockchain networks, there is a risk that this data could be accessed or used without the proper authorization. Additionally, the decentralized nature of blockchain means that data is stored on multiple nodes, which can make it more difficult to control and manage access to it. To give further evidence, this is generally how scammers are able to scams on the crypto end, which uses a blockchain network to safeguard crypto wallets on a whole.
    Another concern is the potential for hacking and malicious attacks on blockchain networks. As blockchain networks become more widely adopted, they may become more attractive targets for hackers, who could potentially access or alter data stored on the network.
    Furthermore, the complexity of blockchain technology and the limited number of experts in the field, means that there is a risk of human errors, which could lead to data breaches or other security incidents.
    In conclusion, while blockchain technology can provide enhanced security for IoT networks and devices, it also poses some risks when it comes to data security. It's important to consider these risks and take steps to mitigate them, such as by implementing proper security measures and access controls, and by continuously monitoring and updating the network. Additionally, it's important to have a team of experts that can ensure the security and integrity of the blockchain network.
\end{itemize}

\end{document}
